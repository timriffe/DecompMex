%% BioMed_Central_Tex_Template_v1.06
%%                                      %
%  bmc_article.tex            ver: 1.06 %
%                                       %

%%IMPORTANT: do not delete the first line of this template
%%It must be present to enable the BMC Submission system to
%%recognise this template!!

%%%%%%%%%%%%%%%%%%%%%%%%%%%%%%%%%%%%%%%%%
%%                                     %%
%%  LaTeX template for BioMed Central  %%
%%     journal article submissions     %%
%%                                     %%
%%          <8 June 2012>              %%
%%                                     %%
%%                                     %%
%%%%%%%%%%%%%%%%%%%%%%%%%%%%%%%%%%%%%%%%%


%%%%%%%%%%%%%%%%%%%%%%%%%%%%%%%%%%%%%%%%%%%%%%%%%%%%%%%%%%%%%%%%%%%%%
%%                                                                 %%
%% For instructions on how to fill out this Tex template           %%
%% document please refer to Readme.html and the instructions for   %%
%% authors page on the biomed central website                      %%
%% http://www.biomedcentral.com/info/authors/                      %%
%%                                                                 %%
%% Please do not use \input{...} to include other tex files.       %%
%% Submit your LaTeX manuscript as one .tex document.              %%
%%                                                                 %%
%% All additional figures and files should be attached             %%
%% separately and not embedded in the \TeX\ document itself.       %%
%%                                                                 %%
%% BioMed Central currently use the MikTex distribution of         %%
%% TeX for Windows) of TeX and LaTeX.  This is available from      %%
%% http://www.miktex.org                                           %%
%%                                                                 %%
%%%%%%%%%%%%%%%%%%%%%%%%%%%%%%%%%%%%%%%%%%%%%%%%%%%%%%%%%%%%%%%%%%%%%

%%% additional documentclass options:
%  [doublespacing]
%  [linenumbers]   - put the line numbers on margins

%%% loading packages, author definitions

%\documentclass[twocolumn]{bmcart}% uncomment this for twocolumn layout and comment line below
\documentclass{bmcart}

%%% Load packages
%\usepackage{amsthm,amsmath}
%\RequirePackage{natbib}
%\RequirePackage[authoryear]{natbib}% uncomment this for author-year bibliography
%\RequirePackage{hyperref}
\usepackage[utf8]{inputenc} %unicode support
\usepackage{graphicx}
%\usepackage[applemac]{inputenc} %applemac support if unicode package fails
%\usepackage[latin1]{inputenc} %UNIX support if unicode package fails


%%%%%%%%%%%%%%%%%%%%%%%%%%%%%%%%%%%%%%%%%%%%%%%%%
%%                                             %%
%%  If you wish to display your graphics for   %%
%%  your own use using includegraphic or       %%
%%  includegraphics, then comment out the      %%
%%  following two lines of code.               %%
%%  NB: These line *must* be included when     %%
%%  submitting to BMC.                         %%
%%  All figure files must be submitted as      %%
%%  separate graphics through the BMC          %%
%%  submission process, not included in the    %%
%%  submitted article.                         %%
%%                                             %%
%%%%%%%%%%%%%%%%%%%%%%%%%%%%%%%%%%%%%%%%%%%%%%%%%


%\def\includegraphic{}
%\def\includegraphics{}



%%% Put your definitions there:
\startlocaldefs
\endlocaldefs


%%% Begin ...
\begin{document}

%%% Start of article front matter
\begin{frontmatter}

\begin{fmbox}
\dochead{Research}

%%%%%%%%%%%%%%%%%%%%%%%%%%%%%%%%%%%%%%%%%%%%%%
%%                                          %%
%% Enter the title of your article here     %%
%%                                          %%
%%%%%%%%%%%%%%%%%%%%%%%%%%%%%%%%%%%%%%%%%%%%%%

\title{Think of a better title }

%%%%%%%%%%%%%%%%%%%%%%%%%%%%%%%%%%%%%%%%%%%%%%
%%                                          %%
%% Enter the authors here                   %%
%%                                          %%
%% Specify information, if available,       %%
%% in the form:                             %%
%%   <key>={<id1>,<id2>}                    %%
%%   <key>=                                 %%
%% Comment or delete the keys which are     %%
%% not used. Repeat \author command as much %%
%% as required.                             %%
%%                                          %%
%%%%%%%%%%%%%%%%%%%%%%%%%%%%%%%%%%%%%%%%%%%%%%

\author[
   addressref={aff1},                   % id's of addresses, e.g. {aff1,aff2}
   corref={aff1},     
   noteref={n1},                  % id of corresponding address, if any
   %noteref={n1},                        % id's of article notes, if any
   email={jmaburto@health.sdu.dk}   % email address
]{\inits{JMA}\fnm{Jos\'e Manuel} \snm{Aburto}}
\author[
   addressref={aff2},
   noteref={n1},
   email={riffe@demogr.mpg.de}
]{\inits{TR}\fnm{Tim} \snm{Riffe}}
\author[
   addressref={aff1},                   % id's of addresses, e.g. {aff1,aff2}
   email={vcanudas@health.sdu.dk}
]{\inits{VCR}\fnm{Vladimir} \snm{Canudas-Romo}}

%%%%%%%%%%%%%%%%%%%%%%%%%%%%%%%%%%%%%%%%%%%%%%
%%                                          %%
%% Enter the authors' addresses here        %%
%%                                          %%
%% Repeat \address commands as much as      %%
%% required.                                %%
%%                                          %%
%%%%%%%%%%%%%%%%%%%%%%%%%%%%%%%%%%%%%%%%%%%%%%

\address[id=aff1]{%                           % unique id
  \orgname{Department of Public Health \& Max Planck Odense Center on the Biodemography of Aging at University of Southern Denmark}, % university, etc
  \street{J.B. Winsl{\o}ws Vej 9},                     %
  \postcode{5000}                                % post or zip code
  \city{Odense},                              % city
  \cny{Denmark}                                    % country
}
\address[id=aff2]{%
  \orgname{Max Planck Institute for Demographic Research},
  \street{ Konrad-Zuse-Stra{\ss}e 1},
  \postcode{18057}
  \city{Rostock},
  \cny{Germany}
}

%%%%%%%%%%%%%%%%%%%%%%%%%%%%%%%%%%%%%%%%%%%%%%
%%                                          %%
%% Enter short notes here                   %%
%%                                          %%
%% Short notes will be after addresses      %%
%% on first page.                           %%
%%                                          %%
%%%%%%%%%%%%%%%%%%%%%%%%%%%%%%%%%%%%%%%%%%%%%%

\begin{artnotes}
%\note{Sample of title note}     % note to the article
\note[id=n1]{Equal contributor} % note, connected to author
\end{artnotes}

\end{fmbox}% comment this for two column layout

%%%%%%%%%%%%%%%%%%%%%%%%%%%%%%%%%%%%%%%%%%%%%%
%%                                          %%
%% The Abstract begins here                 %%
%%                                          %%
%% Please refer to the Instructions for     %%
%% authors on http://www.biomedcentral.com  %%
%% and include the section headings         %%
%% accordingly for your article type.       %%
%%                                          %%
%%%%%%%%%%%%%%%%%%%%%%%%%%%%%%%%%%%%%%%%%%%%%%

\begin{abstractbox}



\begin{abstract} % abstract
\parttitle{Background} The second half of the 20th century was marked by sizable improvements in mortality, living conditions and health in most Latin American countries. In Mexico, these improvements have slowed down recently as a result of opposing trends in particular causes of death. We aim to extend these findings by age groups to the 32 Mexican states by measuring the potential gains in life expectancy due to avoidable causes of death.
\parttitle{Methods} We use mortality data from 1990 to 2015 for all states and calculate temporary life expectancies for three large age groups, and compare these with a low mortality benchmark. We use the concept of avoidable mortality  and use standard decomposition techniques to disentangle age-cause-specific effects on survival.
\parttitle{Results} We find improvements in survival for the population below age 15. However, the adult population aged 15 to 39 shows deterioration among males after 2006 in almost every state as a result of an increase in homicide mortality. Adults aged 40 to 74 show an unexpected decrease in the low mortality benchmark, indicating universal deterioration in both males and females. State-specific departures from this benchmark was caused by ischemic heart diseases, diabetes, cirrhosis and homicide mortality, mainly. We find large health disparities between states, particularly for the adult population and specially after 2005.
\parttitle{Conclusions} Mexico has succeeded in reducing mortality and between-states inequalities in children and the young population. However, the adult population is becoming vulnerable as they have not been able to reduce the burden of conditions amenable to health services and some related to public policies (e.g. homicides). This has led to large health disparities between states in the last 25 years.
%\parttitle{Second part title} %if any
%Text for this section.
\end{abstract}

%%%%%%%%%%%%%%%%%%%%%%%%%%%%%%%%%%%%%%%%%%%%%%
%%                                          %%
%% The keywords begin here                  %%
%%                                          %%
%% Put each keyword in separate \kwd{}.     %%
%%                                          %%
%%%%%%%%%%%%%%%%%%%%%%%%%%%%%%%%%%%%%%%%%%%%%%

%to think about
\begin{keyword}
\kwd{health inequalities}
\kwd{adult health}
\kwd{causes of death}
\kwd{homicides}
\kwd{ischemic heart diseases}
\kwd{diabetes}
\kwd{cirrhosis}
\end{keyword}

% MSC classifications codes, if any
%\begin{keyword}[class=AMS]
%\kwd[Primary ]{}
%\kwd{}
%\kwd[; secondary ]{}
%\end{keyword}

\end{abstractbox}
%
%\end{fmbox}% uncomment this for twcolumn layout

\end{frontmatter}

%%%%%%%%%%%%%%%%%%%%%%%%%%%%%%%%%%%%%%%%%%%%%%
%%                                          %%
%% The Main Body begins here                %%
%%                                          %%
%% Please refer to the instructions for     %%
%% authors on:                              %%
%% http://www.biomedcentral.com/info/authors%%
%% and include the section headings         %%
%% accordingly for your article type.       %%
%%                                          %%
%% See the Results and Discussion section   %%
%% for details on how to create sub-sections%%
%%                                          %%
%% use \cite{...} to cite references        %%
%%  \cite{koon} and                         %%
%%  \cite{oreg,khar,zvai,xjon,schn,pond}    %%
%%  \nocite{smith,marg,hunn,advi,koha,mouse}%%
%%                                          %%
%%%%%%%%%%%%%%%%%%%%%%%%%%%%%%%%%%%%%%%%%%%%%%

%%%%%%%%%%%%%%%%%%%%%%%%% start of article main body
% <put your article body there>

%%%%%%%%%%%%%%%%
%% Background %%

\section*{Background}
The second half of the 20th century was marked by sizable improvements in mortality, living
conditions and health in most Latin American countries \cite{who2000}. 
In Mexico, these improvements have slowed down recently as a result of opposing
trends in particular causes of death. For instance, homicide and diabetes
increased during the first decade of the 2000's, even as infectious and
respiratory diseases continued to fall over the same period. While life
expectancy at birth increased by 4.3 years for males (from 67.6 to 71.9) and 3.4
for females (from 73.8 to 77.2) between 1990 and 2000 \cite{SOMEDE},
between 2000 and 2010, life expectancy at birth entered into a period of
stagnation for males and slowed progress for females \cite{canudas2014}. 


This
period coincides with ongoing public health interventions, such as the Universal Vaccination Program, and with the implementation of Seguro
Popular, which aim to provide primary and secondary
health care to the uninsured population and allocate funds to cover catastrophic
health expenditures \cite{knaul2005}. Further, conditional cash transfer programs were introduced to supply incentives for families to reinvest in education, health, and nutrition in 1997 \cite{neufeld2012}. Some evidence
suggests that Mexico experienced substantial decreases in infant and child
mortality, along with improvements that contributed to the reduction of
mortality and in the prevalence of acute malnutrition between 1980 and 2000
because of these interventions \cite{sepulveda2006}. Similarly, by 2012 Seguro Popular had provided health insurance coverage to an additional 52 million
people in Mexico that previously did not have any access to public health care and, as a result, there has been a reduction in catastrophic health expenditures \cite{knaul2012}.

Conditional cash transfers are focused on the poorest states, and Seguro Popular was introduced at different times in different states. In addition, Mexico faces a rapid aging process in which we can anticipate the interaction between infectious diseases and noncommunicable conditions \cite{Bygbjerg1499} on the adult population.\footnote{The percentage of the population aged 60 or older is projected to go from 10\% in 2015 to 15\% in 2030 \cite{CONAPO}.} Although these actions underscore broad progress in public health interventions, they mask disparities between Mexican states and the epidemiological patterns for different age groups. Therefore, it is necessary to assess the varied impacts that these interventions may have had on mortality in Mexican states \cite{urquieta2015evolution}. 

 
 One approach to approximate the impact of health care and other interventions, and to reveal potential areas of improvement is by operationalizing the
 concept of Avoidable or Amenable Mortality (hereafter abbreviated AM)
 \cite{nolte&mckee2004, nolte&mckee2008,elo2014}. This categorization of mortality aims to measure the quality of health service systems by selecting certain
 causes of death that should not occur in the presence of effective and
 timely health care. Among industrialized countries, such as United States,
 Australia, France, Japan, a reduction in AM rates was
 observed over the past 20 years
 \cite{nolte&mckee2008}. Avoidable mortality rates fell, on average, by 17\%
 for males and 14\% for females in these countries. Despite mortality reductions from cancers and circulatory diseases for
 both sexes, heterogeneity between countries persists, with the United
 States showing the smallest reductions (around 5\%) for both sexes  \cite{nolte&mckee2008}. 
 
 In Mexico, the components of avoidable mortality had different trends since the
late 1990's. Mortality from infectious diseases and nutrition-related conditions decreased between 2000 and 2004 \cite{francomarina2006}, while deaths related to diabetes and circulatory diseases increased in the same period \cite{agudelo2014efecto}. Importantly, increases in the latter causes
of death were concentrated in the poorest states of the country
\cite{davila2014mortalidad}. 

We extend previous analyses by using the most recent available data to study mortality trends by cause of death for all 32 states, by sex, and over the full period from 1990 to 2015. This choice of period covers several public health interventions and captures several major trends in state and cause-of-death variation. We further segment AM into health intervention-related AM and
behavior-related AM causes that capture the epidemiological patterns of Mexico \cite{Aburto2015}. In addition, our work differentiates from earlier studies by comparing state mortality patterns with an easy-to-understand low-mortality benchmark calculated for large age groups (i.e. 0-14, 15-49, 50-84). This concept has been previously used in mortality studies \cite{whelpton1947}, and further developed elsewhere 
\cite{wunsch1975minimum,vallin2008minimum}. Deviations from the low-mortality benchmark indicate a strong potential for improvement.

We hypothesize age-dependent variations in mortality outcomes.
In particular, we expect convergence between states and improvement in survival
for young people, since public health interventions are mainly focused on infant and child health. For instance, the vaccination program and Seguro Popular aim to fully cover children in the entire country, and recent
evidence suggests a decrease in mortality below age 15 due to a decline
in infectious and respiratory diseases \cite{gonzalez2016mexico}. On the contrary, we
expect little improvement in survival for the young-adult population due to the sudden and egregious rise in homicide mortality, and among older adults because of the increase in diabetes along with endocrine/metabolic diseases in these ages in the country \cite{gonzalez2011health,gonzalez2016mexico}. Although every
state has the commitment to provide universal coverage and equitable access to
health care, we anticipate disparities between states
in mortality improvements due to state differences in epidemiological patterns  \cite{gomez2016dissonant} and differences in how  health care programs have been delivered to the population
\cite{Frenk2006}.


\section*{Data \& Methods} 
Our analyses are based on publicly available anonymized datasets. We used microdata death files produced by the
Mexican Statistical Office (INEGI) from 1990 to 2015 \cite{INEGI}. From these data, 
information on causes of death by single age, sex, and state of residence at the
time of death was extracted. Population estimates from 1990 to 2015 came from the Mexican Population Council (CONAPO) \cite{CONAPO}. These estimates adjust for age misstatement, undercounting, and interstate and international migration. Death counts and estimated of the population exposed to risk were used to calculate cause-age-specific death rates by sex and state from 1990 to 2015.

\subsection*{Classification of Causes of Death}
To classify deaths we use the concept of ``Avoidable/Amenable Mortality'' (AM) \cite{nolte&mckee2004, nolte&mckee2008}. We group causes of death into ten categories based on a previous classification  \cite{elo2014} that has recently been adapted to the case of Mexico \cite{Aburto2015}. The first category refers to those conditions that are susceptible to medical intervention, such as infectious and respiratory diseases, and it is labeled ``Causes amenable to medical service''. We separate diabetes, ischemic heart diseases (IHD), HIV/AIDS, lung
cancer, and cirrhosis because these causes are susceptible to both health behavior
and medical service, and because the first two represent major causes of death
in Mexican adults \cite{gomez2016dissonant}. We also separate
homicide, road traffic accidents, and suicide because they have emerged as
leading causes of death among young people, and the first two recently had a sizeable
impact on life expectancy in Mexico \cite{Aburto2015}. Remaining causes were grouped into a single category labeled ``Other causes''. 

Death data was originally classified according to the International Classification of Diseases (ICD), revision 9 for years 1990 to 1997 and revision 10 for 1998 to 2015 (see Additional file 1 Table 1 for details on ICD codes for each category). For the sake of a consistent cause-of-death classification over studied period, we grouped specific causes using codes from a previous study on avoidable mortality in the US \cite{elo2014}. To check the validity of these cause-of-death bridge codes in Mexico, we performed a sensitivity analysis and did not find major ruptures in mortality trends by AM classification (See Additional file 1 figure 1). Although ill-defined causes represent a small percentage of the total deaths (2\% in 1992 \cite{rivera2002epidemiological}), we decided to leave them in the residual category rather than redistribute over other causes of death. We suspect that ill-defined causes could be related to specific conditions, such as homicides, in Mexico.

We truncate analysis at age 85 because cause of death classification and age reporting are considered to be inaccurate in death registration at older ages \cite{tobias2001}. Further, age 85 is higher than both the mean and modal lifespan in Mexico, and most important changes in survival are captured below it. 

%Table~\ref{tab:causes} lists the cause of death categories we use with relative frequencies by sex for the period 1990-2015.

\subsection*{Age Groups}

We calculate life expectancy in three large age groups to capture mortality differences along the lifespan. Life expectancy in each age group simply refers to the average years of life lived between two ages conditional on survival to the lower age bound. The first age group refers to people aged 0-14. This group is likely to represent improvements in causes amenable to medical service (e.g. infectious diseases and conditions of perinatal period) \cite{canudas2014}. The second group, aged 15-49, is used to capture the effect of homicide mortality and external causes historically related to the young-adult mortality hump. This age group had an important impact on changes in state life expectancy in the first decade of the 2000s \cite{Aburto2015}. The third group covers older adults aged 50-84. We focus on adults because they likely represent a vulnerable group due to deterioration in non-communicable diseases and injury-related mortality in recent years \cite{gonzalez2011health,gomez2016dissonant}.

\subsection*{Statistical \& demographic Methods}
We smooth cause-specific death rates over age and time for each
state and sex separately using a 2-d p-spline
to mitigate random variations between ages \cite{GC2012}. We then calculate period life tables up to
age 74 for males and females from 1990 to 2015 following standard demographic methods \cite{HMDMP}. We calculate the temporary life expectancy \cite{arriaga1984} (See Additional file 1 for a technical overview) and estimate cause-specific contributions to the difference between
each state and the low mortality benchmark using
 standard decomposition techniques \cite{horiuchi2008}. Finally, to measure the level of disparities between states over time, we estimate the coefficient of variation and the Gini coefficient on temporary life expectancy within each age-group and year. In addition, we perform a two way ANOVA and post hoc tests to analyze disparities in temporary life expectancy between states and age groups in Mexico. 

\subsection*{Low mortality benchmark}
Our low-mortality benchmark is calculated in the basis of the lowest observed mortality rates by age, cause of death, from among all states for a given sex and year.

The resulting minimum mortality rate schedule has a unique age profile, and it determines our benchmark temporary life expectancy. The minimum mortality schedule can be treated as the best presently achievable mortality assuming perfect diffusion of the best available practices and technologies in Mexico \cite{vallin2008minimum}. This value is a practical reference because it is based neither on a projection of improvements into the future nor on an arbitrary and likely dissimilar population. 


\subsection*{Limitations}
Mortality data from Mexico are
likely to present inaccuracies in cause-of-death classification due to
comorbidities, particularly at older ages \cite{tobias2001}. To mitigate this,
we focus on ages below 85 and grouped causes of death using ICD codes.
Our estimates regarding homicide mortality are likely to be
underestimated due to inaccurate practices regarding counting, reporting,
and due to the large number of missing individuals in Mexico \cite{HRW2011}.

Avoidable mortality should be understood as an indicator of potential
weaknesses with respect to health care and some public health policies and not
as a definitive assessment \cite{nolte&mckee2008}. The amount of deaths that should be considered avoidable within the avoidable classification is not clear \cite{beltran2011avoidable}. For instance, some researchers consider only 50 percent of heart disease mortality to be avoidable \cite{nolte2012amenable, holland2003}. We do not have information to precisely
measure percentages of avoidable mortality within cause groups in Mexico. Nonetheless, the
difference between a given mortality schedule and the best mortality schedule of
the same year can be conceived of as a minimal definition of avoidable
mortality. The benchmark mortality schedule sets a lower bound to how much mortality could have been avoided. Certainly, even the best mortality schedule will contain elements of mortality that
most would consider avoidable. To the extent that the components of the benchmark schedule were indeed
attained somewhere in Mexico, one can view any excess mortality
with respect to the benchmark schedule as avoidable. We believe this perspective improves on the AM concept by giving a directly measurable standard against which to estimate avoidable deaths.

\section*{Results}

\subsection*{Trends in mean survival for Mexican states by age-groups}

Figure \ref{Fig1} presents life expectancy between two ages (temporary life expectancy) by state  for three large age groups, young (ages 0-14), young adults (15-49) and older adults (50-84), over the period 1990-2015. Grey lines refer to each one of the 32 states; and the blue lines represent the low mortality benchmark. The black line at the top of each panel indicates the maximum survival in each age group. For example, the young group has a maximum of 15 livable years, while young and older adults have up to 35 years conditional on surviving to 15 and 50, respectively. Any gap between a state line and the blue line represents potential years of life if mortality were to achieve the low mortality benchmark.

All states show improvements in the young age group from 1990 to the early 2000s approaching the low mortality benchmark, which is very close to full survival below age 15. In contrast, Mexico City (former Federal District) has lagged behind in reducing mortality at these ages; and some southern states, such as Tabasco, Puebla and Chiapas, show the lowest levels of life expectancy below age 15 throughout the entire period.

Life expectancy between ages 15 and 49 shows a common shift after 2005 among males in almost every state in Mexico. In 2005,young males were living on average 33.8 years. By 2010, 21 states were below this number. Chihuahua, Sinaloa and Durango, in the Northern region, experienced substantial losses in this year, and consequently the largest departures from the low mortality benchmark. In 2015, the states with the lowest survival were Guerrero and Tabasco from the Southern part of the country, and Chihuahua, the bordering state with  Texas in the US. The average over states was 33.8, a figure a  bit lower than ten years before. Trends for females are closer to the low mortality benchmark. However, Chihuahua also show a clear downward after 2005.

In older adults, life expectancy between ages 50 and 84 shows stagnation and deterioration during the entire period. Even the low mortality benchmark exhibits a gradual downward trend, pointing to increases in adult mortality in every state. Of a maximum of 35 years, females in the Mexican states were living on average around 28.8 years in 1990 and 28.6 in the beginning of the 2000s. In 2010 the respective value is 28.3, and more recently 28.6 in 2015. Among males, since 1990, the average between states has not passed 26.8 years and has gradually decreased to 26.5 in 2015. As in young males, some states experience a shift after 2005, leading to lowering the average lived years between ages 50 and 84.\\

\begin{center}
[Figure \ref{Fig1} about here]
\end{center}

\subsection*{Health disparities between states and age groups}
Figure \ref{Fig2} shows the trends in inequalities between states in Mexico for three large age groups, as measured by the coefficient of variation (results for the Gini coefficient are reported in Additional file 1 figure 2). These indicators measure the variation in temporary life expectancy between states in different age groups. Larger values are related to higher disparities between states. Both indicators are relative and have the property that even if temporary life expectancy refers to different range of ages, e.g. 0-14 and 15-49, the values are still comparable over age groups and time. 

Since 1990, inequality in life expectancy for the young population has been decreasing. Young adults show even lower values than the population at younger ages in females. However males show a crossover in the beginning of the 2000s and after 2005 the level of disparities begun to increase leading to higher inequality. The highest values are observed in the period 2009-2011. By 2015 the level has not yet recovered, and still is higher than that of the young population.

Older adults show substantially higher inequality than the other age groups in the entire period. Similar to young male adults, after 2005 they experienced an upturn, and a slowly recovery until 2012. From 2013, both males and females show a rise in disparities between states. Importantly, women show less inequality in every age group at any year. 

We further performed a two-way ANOVA on temporary life expectancy by state and age-groups controlling by year. There was a statistically significance interaction between the effects of states and age groups [$F=19.62, p < .001$]. There were statistically significant differences in temporary life expectancy between age groups [$p < .001$] and states [$p < .001$]. Tukey's HSD post hoc tests were carried out. Results show that 74.9\% of 4,560 possible pairs of comparisons were significantly different at the level of $p < .001$. To illustrate these results, Figure 2 shows... \\

\begin{center}
[Figure \ref{Fig2} about here]
\end{center}


\subsection*{Causes-decomposition analysis}


In figures \ref{Fig3} and \ref{Fig4}, the Mexican states in each region are arranged according to potential gains in survival for older adults in 2015, i.e. departure of each state from the low mortality benchmark. 

Figure \ref{Fig3} shows how causes amenable to medical service, diabetes, ischemic heart diseases (IHD), lung cancer, cirrhosis, homicide and road traffic accidents contributed to the gap between each state and the low mortality benchmark from 1990 to 2015 for male older adults (ages 40-84). These are the causes of death that contributed the most to impeding the states from achieving the low mortality benchmark. Light-yellow colors indicate no contributions to the gap, which means that are very close to the low mortality benchmark within each category. Darker red hues indicate larger contributions to the gap. If a particular state is improving during the period, it shows a transition from red to light-yellow. 

Medically amenable causes of death show gradual improvements in most states from 1990 to 2015, leading to decreasing the gap with the benchmark in this category. However, large disparities between states and large room for improvements remain. For example, Baja California, Chihuahua, Sonora and Coahuila from the northern region; along with Mexico City in the central region; and most states in the south show substantial contributions to the gap with the low mortality benchmark. Mortality caused by diabetes has increasingly contributed to the gap with the benchmark in some states. Coahuila and Tamaulipas in the north; Mexico City, Guanajuato and Tlaxcala in the central region; and Puebla, Veracruz and Tabasco in the south, show deterioration in diabetes mortality in the last decade, leading to widening the gap with the benchmark. Similarly, IHD affects significantly the north part of the country, while cirrhosis is mostly concentrated in the south. Although lung cancer and road traffic accidents do not have the same magnitude as these previous causes, in every state these conditions contribute to the gap with the benchmark. Cause specific mortality for homicides increased the gap with the low mortality benchmark in some states affecting ages 50-84. In particular, Chihuahua, Durango and Sinaloa in the north; Colima, Michoacan and Nayarit in the central region; and Morelos and Guerrero in the southern part of the country were affected.

\begin{center}
[Figure \ref{Fig3} about here]
\end{center}

Causes amenable to medical service, diabetes and IHD are the causes of death that mostly contributed to the gap with the low mortality benchmark among older adult females (see Additional file 1 figure 3). Although almost every state shows improvements in causes medically amenable, these conditions still represent potential years of life if the low mortality benchmark were achieved in this age group. Diabetes shows deterioration in recent years in several states, such as Coahuila, Tamaulipas in the north; Guanajuato and Tlaxcala in the central area; and Puebla, Veracruz and Tabasco in the south, among others, widening the gap with the low mortality benchmark. Moreover, IHD contributes significantly to the gap, particularly in the northern region, and there are not clear improvements throughout the entire period 1990-2015.

Among the young population below age 15, improvements in survival and in reducing the gap with the low mortality benchmark were mainly driven by causes amenable to medical service in both females and males. However,  Mexico, Tlaxcala, Mexico City in the central region, along with almost every state from the south could improve in this set of conditions (see Additional file 1 figures 4 and 5). Finally, homicide mortality and road traffic accidents mainly explain the trends in the gap with the benchmark among young male adults (ages 15-49) (see Additional file 1 figures 6 and 7). Homicides contributed more than 2.5 years to the gap with the low mortality benchmark in 2010 in Chihuahua, other states in the north like Baja California, Coahuila, Tamaulipas, Nuevo Le\'on, Durango and Sinaloa experienced substantial increases in this cause of death after 2005 explaining most of the gap with the benchmark. States were also affected in the central and southern regions. Importantly, some states still show high impact of homicides with respect to the benchmark, such as Guerrero and Morelos in the south; Colima and Guanajuato in the central region.
The impact of the remaining  AM categories in ages 0-14 and 15-49 is negligible. \\

\subsection*{Potential gains and causes of death in 2015}

Figure \ref{Fig4} shows the potential gains in survival for male older adults (ages 50-84)  if the low mortality benchmark were achieved in 2015 in the left panel (Results for previous years and other age groups are shown in the Additional file 1). The right-hand panel shows the proportion of potential gains explained by specific causes of death for 2015.

Every state in Mexico could increase survival by at least one year in older adult ages if they were to achieve the low mortality benchmark. However, for 17 of them the gap with the benchmark is higher than 2 years, and for 3 states in the northern region is greater than 3 years. In females, with the exception of Sinaloa and Nayarit, all the states show a gap greater than one year of life between ages 50-84 (see Additional file 1 figure 9). Since 2005, no major improvements were observed, and in some states the potential gains even increased in 2015 compared to 2010, for example San Luis Potos\'i and Zacatecas in the north; Colima, Guanajuato, Mexico City in the central region; and all the southern region, with the exception of Guerrero. 

More than half of these potential gains in life expectancy between 50 and 84 years are due to avoidable causes of death in every state of Mexico (right-hand panel), and in more than 60\% of the sates these causes account for more than 75\% of potential years. The three main causes of death explaining these differences with respect to the low mortality benchmark are conditions amenable to medical service (AMS) (green bar), diabetes (yellow) and IHD (purple). This is true also for females (see Additional file 1 figure 10). Cirrhosis explains a considerable proportion mainly in states in the central and southern region, and the effect of homicide mortality is present in almost every state, particularly in Guerrero, Morelos (south), Nayarit, Colima (center) and Sinaloa in the north. 

\begin{center}
[Figure \ref{Fig4} about here]
\end{center}



\section*{Discussion}
\subsection*{Young survival}

This analysis demonstrates the potential contribution of achieving the low mortality benchmark to improvements in survival. However, it is concerning that the low mortality benchmarks have not been steadily increasing over the period studied. Trends were flat for children, they are experiencing almost full survival before age 15. More worrisome is the common shift after 2005 in adults aged 15-39 and decreasing survival among older adults aged 35-74.

Despite the flattening pattern of the low mortality benchmark in children, our results show that all states in Mexico have improved survival towards this benchmark and to the maximum survival. Causes amenable to medical service are at the heart of such improvements, consistent with decreases in infectious and respiratory diseases associated with public health interventions targeted to children in Mexico previously documented \cite{sepulveda2006}. For example, Puebla and Tlaxcala improved survival over half a year since the 1990's. By 2010 survival was improved so that all states' temporary life expectancy ranged between 14.6 and 14.8 years. We further estimated survival inequalities between states by age group for every year.  Indeed, survival inequality below age 15 was reduced paralleling improvements in survival during the period. In addition, our results are also consistent with advances in coverage for skilled attendance at delivery, which by 2012 remained above 90\% and more than 78\% of children under age one visited the doctor to monitor their development and growth  \cite{urquieta2015evolution}. Moreover, vaccination coverage has been achieved for the entire young population, the success of such public health interventions are in line with our results, underscoring the improvements in survival in the population younger than 15 years associated to the progress detected in health insurance coverage due to vaccination programs and the implementation of the Seguro Popular \cite{urquieta2015evolution}. Although average years lived below 15 has improved, there still exist areas of opportunity to achieve full-survival under age 15 in causes amenable to medical service, mainly in states in the Central and Southern regions of the country. 


\subsection*{Young adults}

Young adults (ages between 15 and 39) show a converging pattern towards the low mortality benchmark in all states just until 2005. A sudden increase in homicide rates widened the gap with the low mortality benchmark by almost four times on average in 2010, relative to the level observed in 2005. Previous research documented losses in the overall life expectancy up to three years in the state of Chihuahua (the bordering state with Texas, USA) and almost two years in Sinaloa, Durango (North) and Guerrero (South) between 2005 and 2010 due to homicides \cite{Aburto2015}. Our findings show that the trend towards the low mortality benchmark was reversed after 2005 due to the increase in homicide mortality, with a peak in 2010-2011. Although homicide rates decreased after 2011, they still are the main cause of death contributing to the gap between the observed survival and the low mortality benchmark in particular states, such as Sinaloa, Durango in the North, Nayarit and Michoac\'an in the cetral region, and Guerrero in the South. These findings underscore the need for effective interventions to reduce homicide mortality, as it still contributes the most to survival shortcomings among the young-adult population and mortality inequality among states. Even ten years after the national security strategy that aimed at reducing drug cartels' operations started and homicides begun to spread all over the country \cite{espinal2015analysis}, the effect of homicide on average survival is appalling. Between-state inequality in female survival was much smaller over the same period.


\subsection*{Older adults}

In Mexico, since the beginning of the 1990's, adult survival in ages 40-74 deteriorate for males and stagnate for females. Our results help explain on this pattern showing that the low mortality benchmark decreased as a result of state-specific mortality trends and the interaction between specific causes of death. In particular, there are offsetting effects between improvements in causes amenable to medical service, such as infectious and respiratory diseases, and deterioration in diabetes, isquemic heart diseases (IHD), and behavior-related mortality through cirrhosis and homicides. 

Out of 35 potential years, adult females in Mexico are living less than 33 and males less than 31 since the 1990's. The increase in  diabetes, IHD and cirrhosis mortality is at the heart of survival's deterioration, with clear regional variations. Although improvements in causes amenable to medical service were witnessed, almost every state still has potential to improve in this ages, in particular the Northern states of Sonora, Chihuahua and Baja California. Diabetes mortality increased over the period and contributed to increases in the gap to achieve the low mortality benchmark. Diabetes-related mortality increased 23\% from 1998 to 2002, and the prevalence of diabetes was estimated at 14.4\% in the adult population in 2006. These figures underscore the emerging epidemic of diabetes \cite{glassman2010confronting}. To put this in perspective, Coahuila, the state of Mexico, Guanajuato, the Federal District, Tabasco and Puebla could increase survival by almost one year if diabetes mortality were to achieve the low mortality benchmark. Similarly, mortality related to IHD contributes to lowering life expectancy in adults. There is a clear regional pattern in the country. Almost all the states in the Northern region could potentially benefit with one additional year in life expectancy if the low mortality benchmark were reached, where as the Central and Southern regions present a lower impact of IHD. Cirrhosis-related mortality shows a higher impact in the Southern and Central states of the country, particularly in Quer\'etaro, M\'exico state, Hidalgo (central area) and Puebla and Oaxaca in the South. Both diabetes and IHD mortality are closely related to obesity prevalence, previous research anticipated that the increasing levels of obesity in Mexico could compromise gains in life expectancy \cite{monteverde2010obesity}. These regional differences on cause-specific mortality led to increases in health inequalities in adults aged 30-74 after 2006 for males and stagnation among females (figure \ref{fig:Gini}). 


There is still potential for improvements to reduce state-mortality differences and improve the survival among the adult population in Mexico. Several screening and prevention strategies (e.g. PREVENIMSS) for early diabetes and hypertension have been implemented in the country. However, as previous research has found, they are far from achieving the ultimate goal and including the entire population \cite{castro2010potential}. In addition, \cite{behrman2013health} show that the conditional cash transfer program PROSPERA improves health significantly for adult women older than 50. The authors also noted that the effect on men's health is much lower. They argue that this could be the result of the lack of inclusion of men in the program and the main role of women in the program's requirements. Women are recipients of the monetary transfers and they are more likely to attend clinic visits and follow health measures given by doctors in these clinics than men.\\


\section*{Conclusion}

Health inequalities have been previously identified as a current challenge for Mexico \cite{gomez2016dissonant,gutierrez2016health}, here we give cause and age-specific targets to improve longevity and at the same time reduce inequalities between states. Since different risk factors are present at different ages for different causes of death in each state, we quantify potential gains in survival of achieving a low mortality benchmark conditioned to surviving at different ages in all 32 states by sex. This allowed us to identify three main stories over the last 25 years. Firstly, improvements in causes amenable to medical service, such as infectious \& respiratory diseases and birth conditions, led to achieve almost full survival below age 15 for the 32 states. 

The increase in homicide mortality reversed gains in male life expectancy, particularly between ages 15-50. These results are consistent with previous studies quantifying the effect of homicide mortality on the stagnation of national life expectancy in the first decade of the 21st century \cite{canudas2014}, and with the reversal experienced in average length of life in most states between 2005 and 2010 \cite{Aburto2015}. Our results extend such studies by adding five years of information. We found that after ten years of the unexpected rise in homicide mortality, most states have experienced a slow recovery after 2010. However, the impact of homicide is still higher than the levels observed in 2005. Between 2010 and 2015, this cause of death accounted for most departures in life expectancy between ages 15 and 50 with the low mortality benchmark. The states that show higher impact in 2015 are Baja California, Chihuahua and Sinaloa in the northern region; Colima, Jalisco, Guanajuato, Mexico state and Michoac\'an in the central region show a recent increase in homicides that led to widening the gap with the low mortality benchmark; and Guerrero is the most affected in the southern region. To put these recent wave of homicides in perspective, Guerrero would achieve an extra year of life between ages 15 and 49 if they get to the lowest level of homicide mortality observed in the country. Importantly, these figures are likely to be underestimates caused by miss-registration of homicides and all missing individuals in the country, particularly in Guerrero \cite{HRW2011,wright2017epistemological}.The disproportionate number of homicides witnessed in the last ten years, along with the increased perceived vulnerability associated to violence, has affected the Mexican population beyond survival. For instance, women are expected to live over 70\% of their remaining life expectancy at age 20 with fear if the perception of vulnerability remains at the 2014 level  \cite{canudas2017Mexico}. There is no simple way to lessen the impact of homicide mortality, but it is clear that the government has not been able to reduce its burden on the population over the last ten years. 




and the stagnation and worsening in mortality on ages above 50

 and identify the varied impact of different causes of death over the life course conditional on survival 

We showed that conditions amenable to health behavior and healthcare, such as diabetes, IHD and lung cancer, are having a sizable impact in determining why mexican states are not improving life expectancy after age 50 in the last 25 years. As a consequence of the very variety of epidemiological patterns between states and over the life course, there are discordant patterns of mortality over age groups and regions. These results point towards strengthening preventive care among the adult population to improve healthy behaviors...

Identifying specific opportunities to improve and put forward solutions to reduce the gap of  the unequal impact of public health interventions on health is a necessary step to promote equitable increases in survival among the Mexican population.% the high degree of social and health inequalities present in the country \cite{Frenk2006} 
 In addition,  given the improvements in health care coverage, the strong role %of institutions, and ongoing public
 %health interventions, 
 
Improving health is a priority for governments of many developing countries. In part to reduce child mortality, improve maternal health and lessen the impact of other infectious diseases, such as HIV/AIDS, to achieve the Millennium Development Goals established for 2015 \cite{united2009millennium}.  Mexico has succeeded in reducing mortality and inequalities in children and the young population. Nevertheless, our results show that older adults are becoming a vulnerable group, and more efforts are required to reduce the burden of conditions amenable to health services and policy-related conditions. In particular, this group lacks comprehensive interventions to reduce the burden of violence through homicides, chronic-degenerative causes of death, such as diabetes and IHD, and behavior-related conditions such as cirrhosis.  

There is no simple way to lessen the impact of such conditions, but it is clear that new approaches are needed to improve survival in the adulthood and to minimize health disparities between states. Preventing diabetes and IHD implies fundamental political challenges. Therefore,  public health initiatives should focus in health care for chronic conditions as recently suggested by \cite{knaul2015achieving}, but they should also influence the population towards improving health behavior. Our results reinforce the need of such, among others public health interventions, with an special focus on older adults in the Mexico. 

%\nocite{oreg,schn,pond,smith,marg,hunn,advi,koha,mouse}

%%%%%%%%%%%%%%%%%%%%%%%%%%%%%%%%%%%%%%%%%%%%%%
%%                                          %%
%% Backmatter begins here                   %%
%%                                          %%
%%%%%%%%%%%%%%%%%%%%%%%%%%%%%%%%%%%%%%%%%%%%%%

\begin{backmatter}

\section*{Competing interests}
  The authors declare that they have no competing interests.

\section*{Author's contributions}
TR and JMA conceptualized the study and performed the demographic and statistical analyses. JMA, TR and VCR helped to interpret the results. JMA and TR wrote the first manuscript and VCR reviewed and revised the manuscript. All authors approved of the final version.
 
\section*{Acknowledgements}
The authors thank their supporting institutions, the European Doctoral School of Demography at Sapienza University, and grants:... The authors also express gratitude to Shammi Luhar for comments to an earlier version of this manuscript.
%%%%%%%%%%%%%%%%%%%%%%%%%%%%%%%%%%%%%%%%%%%%%%%%%%%%%%%%%%%%%
%%                  The Bibliography                       %%
%%                                                         %%
%%  Bmc_mathpys.bst  will be used to                       %%
%%  create a .BBL file for submission.                     %%
%%  After submission of the .TEX file,                     %%
%%  you will be prompted to submit your .BBL file.         %%
%%                                                         %%
%%                                                         %%
%%  Note that the displayed Bibliography will not          %%
%%  necessarily be rendered by Latex exactly as specified  %%
%%  in the online Instructions for Authors.                %%
%%                                                         %%
%%%%%%%%%%%%%%%%%%%%%%%%%%%%%%%%%%%%%%%%%%%%%%%%%%%%%%%%%%%%%

% if your bibliography is in bibtex format, use those commands:
\bibliographystyle{bmc-mathphys} % Style BST file (bmc-mathphys, vancouver, spbasic).
\bibliography{AburtoRiffe_Bib}      % Bibliography file (usually '*.bib' )
% for author-year bibliography (bmc-mathphys or spbasic)
% a) write to bib file (bmc-mathphys only)
% @settings{label, options="nameyear"}
% b) uncomment next line
%\nocite{label}

% or include bibliography directly:
% \begin{thebibliography}
% \bibitem{b1}
% \end{thebibliography}

%%%%%%%%%%%%%%%%%%%%%%%%%%%%%%%%%%%
%%                               %%
%% Figures                       %%
%%                               %%
%% NB: this is for captions and  %%
%% Titles. All graphics must be  %%
%% submitted separately and NOT  %%
%% included in the Tex document  %%
%%                               %%
%%%%%%%%%%%%%%%%%%%%%%%%%%%%%%%%%%%

%%
%% Do not use \listoffigures as most will included as separate files

\section*{Figures}

\begin{figure}[h!]
\centering
\caption{State-specific average survival (grey lines) and low mortality benchmark (blue line) for young (0-14), young adults (15-39) and older adults (40-74) by sex, 1990-2015.}
\label{Fig1}

%\includegraphics[scale=.35]{Figure1_1.pdf}

Note: Y-axis are not in the same scale in order to capture major trends over the period. Source: calculations based on INEGI files. 
\end{figure}



\begin{figure}[h!]
\centering
\caption{Survival inequality (coefficient of variation) by age group and sex, 1990-2010.}
\label{Fig2}
%\includegraphics[scale=.4]{CVfig_1.pdf}

Source: calculations based on INEGI files. 
\end{figure}



\begin{figure}[h!]
\centering
\caption{Cause-specific contributions to the gap between state survival and low mortality benchmark for older male adults, 1990-2015.}
\label{Fig3}

%\includegraphics[scale=.23]{AdultMaleheatmap.pdf}

Source: calculations based on INEGI files. 
\end{figure}


\begin{figure}[h!]
\centering
\caption{Left panel: Potential years gained if benchmark were achieved for older adult males in 2005,2010,2015. Right panel: Proportion of potential years explained by cause of death in 2015 }
\label{Fig4}
%\includegraphics[scale=.45]{Figure_4v2.pdf}

Source: calculations based on INEGI files. 
\end{figure}

%%%%%%%%%%%%%%%%%%%%%%%%%%%%%%%%%%%
%%                               %%
%% Tables                        %%
%%                               %%
%%%%%%%%%%%%%%%%%%%%%%%%%%%%%%%%%%%

%% Use of \listoftables is discouraged.
%%
%\section*{Tables}
% latex table generated in R 3.3.1 by xtable 1.8-2 package
% Wed Jan 18 14:53:15 2017
%\begin{table}[ht]
%\centering
%\caption{Avoidable Mortality classification, 
%             with crude percentages below age 75, years 1990-2015. Source: INEGI files.} 
%\label{tab:causes}
%\begin{tabular}{llllll}
%  \hline
%& Category &\% & Males &  \% & Females \\ 
% &&& (1,000's) & & (1,000's)\\ 
%  \hline
%1&  Causes amenable to medical service & 30.9 \% & 127.5 & 28.2 \% & 114.4 \\ 
%2&  Diabetes & 4.5 \% & 18.7 & 4.9 \% & 19.9 \\ 
%3&  Ischemic heart diseases & 4 \% & 16.4 & 4.2 \% & 17 \\ 
%4&  HIV/AIDS & 0.4 \% & 1.5 & 0.5 \% & 2.1 \\ 
%5&  Lung cancer & 0.9 \% & 3.7 & 0.9 \% & 3.8 \\ 
%6&  Cirrhosis & 2.3 \% & 9.5 & 2.5 \% & 10 \\ 
%7&  Homicide & 2.7 \% & 11.2 & 3 \% & 12.4 \\ 
%8&  Road traffic accidents & 2.8 \% & 11.4 & 2.9 \% & 11.8 \\ 
%9&  Suicide & 0.4 \% & 1.7 & 0.5 \% & 1.9 \\ 
%%10&  Other causes & 24.8 \% & 102.4 & 25 \% & 101.2 \\ 
%   \hline
%\end{tabular}
%\end{table}

%%%%%%%%%%%%%%%%%%%%%%%%%%%%%%%%%%%
%%                               %%
%% Additional Files              %%
%%                               %%
%%%%%%%%%%%%%%%%%%%%%%%%%%%%%%%%%%%

\section*{Additional Files}
  \subsection*{Additional file 1 --- Supplemental material}
   This might refer to a multi-page table or a figure.

  \subsection*{Additional file 2 --- Results}
 Rdata file with all results.


\end{backmatter}
\end{document}
